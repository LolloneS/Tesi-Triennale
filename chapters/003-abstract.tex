\cleardoublepage
\begingroup
\let\clearpage\endgroup
\null\vspace{\stretch{1}}

\chapter*{\centering Abstract}
Le reti Bayesiane sono modelli grafico-probabilistici utilizzati per effettuare inferenze, date delle evidenze. 
Questo lavoro di stage studia come sia possibile la \textit{generazione} di questi modelli \textit{a partire da ontologie}, vale a dire rappresentazioni dettagliate di domini di interesse.

Per cominciare, viene presentato il problema della traduzione in rete Bayesiana di un'ontologia. Vengono inoltre introdotti i concetti teorici fondamentali per la comprensione del lavoro svolto.

L'elaborato procede con un'analisi approfondita della letteratura riguardo al tema trattato, dalla quale emergono sfide ambiziose da affrontare e molti problemi complessi che richiedono ulteriori sforzi di ricerca. Sfide e problemi sono presentati tramite una lettura critica delle pubblicazioni scientifiche relative all'attuale stato dell'arte.

Viene inoltre proposto un approccio che utilizza strumenti pre-esistenti e li coordina in una pipeline semi-automatica per trasformare un'ontologia in una rete Bayesiana, sulla quale condurre processi inferenziali. 
Alcuni esempi d'applicazione di questa metodologia vengono presentati e discussi al fine di dimostrarne l'effettiva validità e di mettere in luce pregi e difetti di questo approccio.


%Bayesian networks are probabilistic graphical models used to make inferences, given some evidence. This work studies how Bayesian networks could be generated starting from ontologies, which are formal representations of domains of interest.
%First of all, the state of the art on the problem of generating Bayesian networks from ontologies is explained. Fundamental theoretical concepts are also introduced in order to give the readers everything they need to understand this work.
%A detailed analysis of the papers on the subject is made, from which emerge many tough challenges and problems that will need to be addressed in further research. Challenges and problems are explained through a critical analysis of the papers mentioned above, which include the state of the art on the process.
%A semi-automatic pipeline to generate a Bayesian network from an ontology and make inferences on it using preexistent tools is suggested. Some examples are presented and discussed in order to show the validity of the procedure and focus on advantages and disadvantages of this approach.


\vspace{\stretch{2}} \null
