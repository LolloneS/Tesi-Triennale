\newpage
\cleardoublepage
\begingroup
\let\clearpage\endgroup
\null\vspace{\stretch{1}}
\chapter*{\centering Prefazione}

Nell'era delle \textit{buzzword}, attribuire il giusto peso a locuzioni come ``intelligenza artificiale'' è  un'operazione tutt'altro che banale. 
Da un lato, si ha un'impressione di (quasi) onnipotenza data da un'arcinota ricchezza di \textit{dati}, custodi di segreti ed informazioni tutti da scoprire; dall'altro, più approfonditamente si affronta l'argomento, più ci si rende conto delle difficoltà, delle sfumature, delle sfide intrinseche che dedicarsi al vastissimo mondo dell'analisi dati e dell'intelligenza artificiale comporta.

Chi avrebbe mai pensato che unire tecnologie semantiche, statistica e grafi avrebbe potuto produrre un ``dottore delle mele''? Personalmente non avrei mai concepito un'idea così curiosa, ed è per questo che, quando il prof. Stella mi ha messo a conoscenza del progetto, ho colto la sfida con grande interesse.

\vspace{\stretch{2}} \null
