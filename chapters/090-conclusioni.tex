\chapter{Conclusioni e prospettive future}

L'approccio proposto risulta essere un valido supporto per la generazione di un sistema diagnostico.
Strutturare la conoscenza sotto forma di ontologia si conferma essere un valido strumento per avere una base di partenza solida e ben organizzata. 

La generazione automatica della rete, però, non può essere considerata uno strumento in grado di risolvere da solo un problema così complesso: è più sensato inquadrarla come un aiuto all'esperto di dominio nella creazione e modellazione del sistema diagnostico. 

Un appunto va fatto proprio sulla figura dell'esperto: il procedimento esposto non mira a sostituirlo, ma ad aiutarlo. Pensare di sostituire le sue conoscenze con una deduzione completamente automatizzata sarebbe insensato, soprattutto nel momento in cui si provasse a passare da una rappresentazione descrittiva (l'ontologia) ad una causale (la rete Bayesiana).
\\
In futuro, il lavoro dovrà concentrarsi su molti aspetti, tra i quali sicuramente:
\begin{itemize}
	\item porre attenzione sulle relazioni causali all'interno dell'ontologia e trovare il giusto modo di rappresentare queste relazioni, al fine di raggiungere una semantica più ricca del limitante \texttt{dependsOn}
	\item cercare di ridurre le responsabilità dell'esperto di dominio, aiutandolo in ogni fase dello sviluppo 
	\item definire chiaramente il valore ``numerico'' di determinate entità (nel caso specifico, \textit{Region}, \textit{Apple\_Transportation}, ...)
	\item trovare il modo migliore per mantenere la correttezza formale richiesta dalle ontologie, evitando un eccessivo \textit{overhead} dovuto ad un numero troppo elevato di classi astratte, superclassi, ...
\end{itemize}

Gli esperti di gestione della conoscenza che curano la Apple Ontology stanno già lavorando a molti di questi punti, che sicuramente richiederanno sforzi di ricerca non indifferenti e prolungati nel tempo. 

Fortunatamente, lo ``stato dei lavori'' lascia spazio a numerosi sviluppi ed a svariate possibilità di ricerca, che faranno emergere molti nuovi aspetti sui quali lavorare e porteranno a miglioramenti che ora, probabilmente, fatichiamo solo ad immaginare.
